К заполнению БД тестовыми данными было решено подойти с изыском и использованием возможностей SQL\@.

Вот пример команды по вставке 300 клиентов, использующий ветвление с CASE и арифметических операций,
конструкций с ARRAY, функциями RANDOM, LPAD, FLOOR, LOWER, REPLACE, встроенной функцией `generate\_series`
и объявленной `transliterate`, конкатенацией для конструирования значений, а также приведение типов через `::`.

\begin{lstlisting}[language=SQL, frame=single, basicstyle=\small\ttfamily, breaklines=true,label={lst:examplecustomer}]
INSERT INTO customer (id, name, email, phone, loyalty_status, bonus_points)
SELECT
id,
CASE
WHEN id BETWEEN 1 AND 100 THEN
(ARRAY['Алексей', 'Иван', 'Пётр', 'Дмитрий', 'Сергей', 'Артём', 'Егор', 'Михаил', 'Олег', 'Владимир'])[(id - 1) % 10 + 1]
...
END,
LOWER(REPLACE(business.transliterate(
CASE
WHEN id BETWEEN 1 AND 100 THEN
(ARRAY['Алексей', 'Иван', 'Пётр', 'Дмитрий', 'Сергей', 'Артём', 'Егор', 'Михаил', 'Олег', 'Владимир'])[(id - 1) % 10 + 1]
...
END
), ' ', '')) || '@lapland.ru',
'+79' || LPAD(FLOOR(RANDOM() * 100)::TEXT, 2, '0') || LPAD(id::TEXT, 3, '0') || LPAD(FLOOR(RANDOM() * 10000)::TEXT, 4, '0'),
(ARRAY['BRONZE'::business.loyalty_status, 'SILVER'::business.loyalty_status, 'GOLD'::business.loyalty_status])[FLOOR(RANDOM() * 3 + 1)],
FLOOR(RANDOM() * 1000)
FROM generate_series(1, 300) AS id;
\end{lstlisting}