\documentclass[a4paper,12pt]{article}
\usepackage[T2A]{fontenc}
\usepackage[utf8]{inputenc}
\usepackage[russian]{babel}
\usepackage{graphicx}
\usepackage{amsmath, amssymb}
\usepackage{geometry}
\usepackage{listings}
\usepackage{tikz}
\usepackage[hidelinks]{hyperref}

\usetikzlibrary{positioning}

\geometry{left=2cm, right=2cm, top=2cm, bottom=2cm}

\title{Учебный проект: БД зоомагазина "Лап-Ландия" и её интересное окружение}

\author{Александр Гоппе}
\date{}


\begin{document}

    \maketitle

    \begin{center}
        \includegraphics[width=0.7\textwidth]{title_good.jpg} % Вставка изображения
    \end{center}

    \vfill
    \begin{center}
        \date{\today}
    \end{center}
    \newpage

    \tableofcontents
    \newpage


    \section{Описание проекта}


    \indent
    Учебный проект \textbf{Лап-Ландия} посвящён построению базы данных для зоомагазина с развёрнутой инфраструктурой,
    включающей реплицируемую БД, балансировщик нагрузки,
    инструменты мониторинга и анализатора данных.
    Демонстрационный стенд разворачивается ``по клику`` через docker compose.

    \subsection{Общая структура}

    \begin{itemize}
        \item \textbf{База данных}: кластер Patroni из трёх узлов.
        \item \textbf{Балансировка}: HAProxy (1 демонстрационный экземпляр).
        \item \textbf{Zookeeper}: 3 экземпляра.
        \item \textbf{Инициализация БД}: мини-контейнер db-script-runner (создаёт базу и юзеров).
        \item \textbf{Миграции}: Flyway.
        \item \textbf{Мониторинг}: PostgreSQL Exporter + Prometheus + Grafana.
        \item \textbf{BI}: Apache Superset.
        \item \textbf{Демонстрационное приложение}: Spring Boot (Java).
        \item \textbf{Демонстрационное тестирование API}: контейнер curl\_runner (отправляет запросы в приложение).
        \item \textbf{Фронт (бонус)}: TypeScript (в разработке).
        \item \textbf{Логирование в ClickHouse (бонус)}: через Logstash.
    \end{itemize}

    Один экземпляр балансировщика HAProxy был развёрнут с пониманием того, что в реальном производстве их должно быть
    минимум два.
    Но, т.к. всё окружение разворачивается в тестовой среде на персональном компьютере ученика и точка отказа, так или
    иначе, одна - работает один экземпляр.
    Теоретически можно было бы обойтись двумя узлами Zookeeper и Patroni, но, как и многое в этом проекте, эти настройки
    были взяты из открытых источников и, ввиду и без того не самой тривиальной архитектуры стенда, было решено
    не тратить время на переделку готовых наработок коллег-ремесленников.

    Аналогично не реплицировались Superset, ClickHouse и другие системы, т.к. предпочтительной задачей в контексте курса
    была выбрана настройка некоторого одного ``целевого`` кластера СУБД.

    \newpage

    \subsection{Диаграмма инфраструктуры}

    \begin{figure}[h]
        \centering
        \resizebox{\textwidth}{!}{%
            \begin{tikzpicture}

                \node (app) [draw, rectangle, rounded corners, minimum width=3cm, minimum height=1cm] {Java-приложение};
                \node (haproxy) [draw, rectangle, rounded corners, minimum width=3cm, minimum height=1cm, below=of app] {HAProxy};
                \node (db1) [draw, rectangle, rounded corners, minimum width=3cm, minimum height=1cm, below left=of haproxy] {Patroni 1};
                \node (db2) [draw, rectangle, rounded corners, minimum width=3cm, minimum height=1cm, below=of haproxy] {Patroni 2};
                \node (db3) [draw, rectangle, rounded corners, minimum width=3cm, minimum height=1cm, below right=of haproxy] {Patroni 3};
                \node (zookeeper) [draw, rectangle, below=of db2, minimum width=3cm, minimum height=1cm] {Zookeeper (3 экз.)};
                \node (superset) [draw, rectangle, rounded corners, minimum width=3cm, minimum height=1cm, right=of haproxy] {Superset};
                \node (logstash) [draw, rectangle, rounded corners, minimum width=3cm, minimum height=1cm, above=of app] {LogStash};
                \node (front) [draw, rectangle, rounded corners, minimum width=3cm, minimum height=1cm, right=of app] {Front/Curl};
                \node (clickhouse) [draw, rectangle, rounded corners, minimum width=3cm, minimum height=1cm, above right=of app] {ClickHouse};
                \node (pgexporter) [draw, rectangle, rounded corners, minimum width=3cm, minimum height=1cm, left=of haproxy] {PG-exporter};
                \node (prometheus) [draw, rectangle, rounded corners, minimum width=3cm, minimum height=1cm, above=of pgexporter] {Prometheus};
                \node (grafana) [draw, rectangle, rounded corners, minimum width=3cm, minimum height=1cm, above=of prometheus] {Grafana};

                \draw[->] (app) -- (haproxy);
                \draw[->] (superset) -- (haproxy);
                \draw[->] (haproxy) -- (db1);
                \draw[->] (haproxy) -- (db2);
                \draw[->] (haproxy) -- (db3);
                \draw[->] (zookeeper) -- (db1);
                \draw[->] (zookeeper) -- (db2);
                \draw[->] (zookeeper) -- (db3);
                \draw[->] (logstash) -- (clickhouse);
                \draw[->] (app) -- (logstash);
                \draw[->] (front) -- (app);
                \draw[->] (pgexporter) -- (haproxy);
                \draw[->] (prometheus) -- (pgexporter);
                \draw[->] (grafana) -- (prometheus);
            \end{tikzpicture}
        }
        \caption{Схема инфраструктуры проекта}
    \end{figure}

    \section{Структура и элементы хранилища данных}

    \subsection{Схемы}
    В базе созданы следующие схемы:
    \begin{itemize}
        \item \textbf{archive}: архивированные данные, для разгрузки оперативной схемы.
        \item \textbf{business}: оперативная схема с данными о покупках в сети зоомагазинов.
        \item \textbf{cron}: техническая схема для элементов расширения cron.
        \item \textbf{migrations}: техническая схема инструмента миграции flyway.
        \item \textbf{public}: общая начальная схема PostgreSQL.
    \end{itemize}

    \subsection{Схема archive: описание}
    Архивируются только данные о покупках, как наиболее тяжёлые и интенсивные.
    Структура и индексы полностью дублируют прототипы из бизнес-схемы.

    \subsection{Схема business: описание}

    Оперативные бизнес-данные о покупках, магазинах, покупателях.

    \begin{figure}[h]
        \centering
        \resizebox{\textwidth}{!}{%
            \begin{tikzpicture}[
                node distance=2cm and 3cm,
                every node/.style={draw, rectangle, rounded corners, minimum width=3cm, minimum height=1cm, align=center}
            ]
                % Определяем таблицы
                \node (category) {category};
                \node (product) [below=of category] {product};
                \node (inventory) [left=of product] {inventory};
                \node (shop) [below=of inventory] {shop};
                \node (supplier) [right=of product] {supplier};
                \node (purchase) [below=of product] {purchase};
                \node (purchase_item) [right=of purchase] {purchase\_item};
                \node (customer) [below=of purchase] {customer};

                % Связи между таблицами
                \draw[->] (product) -- (category);
                \draw[->] (product) -- (supplier);
                \draw[->] (inventory) -- (shop);
                \draw[->] (inventory) -- (product);
                \draw[->] (purchase) -- (customer);
                \draw[->] (purchase) -- (shop);
                \draw[->] (purchase_item) -- (purchase);
                \draw[->] (purchase_item) -- (product);
            \end{tikzpicture}
        }
        \caption{Схема связей таблиц}
    \end{figure}

    \subsubsection{Таблица category}

    \begin
        \normalsize
        \renewcommand{\arraystretch}{1.5} % Регулируем высоту строк
        \begin{tabular}{|l|l|l|}
            \hline
            \textbf{Имя столбца} & \textbf{Тип} & \textbf{Описание}                  \\
            \hline
            id                   & SERIAL       & Уникальный идентификатор категории \\
            \hline
            name                 & TEXT         & Название категории                 \\
            \hline
        \end{tabular}
    \end

    \subsubsection{Таблица product}

    \begin
        \normalsize
        \renewcommand{\arraystretch}{1.5} % Регулируем высоту строк
        \begin{tabular}{|l|l|l|}
            \hline
            \textbf{Имя столбца} & \textbf{Тип}  & \textbf{Описание}                      \\
            \hline
            id                   & SERIAL        & Уникальный идентификатор продукта      \\
            \hline
            name                 & TEXT          & Название продукта                      \\
            \hline
            description          & TEXT          & Описание продукта                      \\
            \hline
            category\_id         & INT           & Ссылка на категорию (category.id)      \\
            \hline
            price                & NUMERIC(10,2) & Цена продукта                          \\
            \hline
            characteristics      & JSONB         & Характеристики продукта в формате JSON \\
            \hline
        \end{tabular}
    \end

    \subsubsection{Таблица shop}

    \begin
        \normalsize
        \renewcommand{\arraystretch}{1.5} % Регулируем высоту строк
        \begin{tabular}{|l|l|l|}
            \hline
            \textbf{Имя столбца} & \textbf{Тип} & \textbf{Описание}                 \\
            \hline
            id                   & SERIAL       & Уникальный идентификатор магазина \\
            \hline
            name                 & TEXT         & Название магазина                 \\
            \hline
            location             & TEXT         & Местоположение магазина           \\
            \hline
        \end{tabular}
    \end

    \subsubsection{Таблица inventory}

    \begin
        \normalsize
        \renewcommand{\arraystretch}{1.5} % Регулируем высоту строк
        \begin{tabular}{|l|l|l|}
            \hline
            \textbf{Имя столбца} & \textbf{Тип} & \textbf{Описание}              \\
            \hline
            shop\_id             & INT          & Ссылка на магазин (shop.id)    \\
            \hline
            product\_id          & INT          & Ссылка на продукт (product.id) \\
            \hline
            quantity             & INT          & Количество товара в магазине   \\
            \hline
        \end{tabular}
    \end

    \subsubsection{Таблица supplier}

    \begin
        \normalsize
        \renewcommand{\arraystretch}{1.5} % Регулируем высоту строк
        \begin{tabular}{|l|l|l|}
            \hline
            \textbf{Имя столбца} & \textbf{Тип} & \textbf{Описание}                   \\
            \hline
            id                   & SERIAL       & Уникальный идентификатор поставщика \\
            \hline
            name                 & TEXT         & Название поставщика                 \\
            \hline
            contact\_info        & TEXT         & Контактная информация поставщика    \\
            \hline
        \end{tabular}
    \end

    \subsubsection{Таблица customer}

    \begin
        \normalsize
        \renewcommand{\arraystretch}{1.5} % Регулируем высоту строк
        \begin{tabular}{|l|l|l|}
            \hline
            \textbf{Имя столбца} & \textbf{Тип}    & \textbf{Описание}                  \\
            \hline
            id                   & SERIAL          & Уникальный идентификатор клиента   \\
            \hline
            phone                & phone\_domain   & Телефон клиента                    \\
            \hline
            email                & email\_domain   & Электронная почта клиента          \\
            \hline
            name                 & TEXT            & Имя клиента                        \\
            \hline
            loyalty\_status      & loyalty\_status & Статус лояльности клиента          \\
            \hline
            bonus\_points        & NUMERIC(10,2)   & Количество бонусных баллов клиента \\
            \hline
        \end{tabular}
    \end
    Несмотря на предполагаемую валидацию на бэкенде, правильные базовые типы в целом в базе не помешают.
    Перечисление поможет избежать случайных описок и логически ограничит значения.

    \subsubsection{Таблица purchase}

    \begin
        \normalsize
        \renewcommand{\arraystretch}{1.5} % Регулируем высоту строк
        \begin{tabular}{|l|l|l|}
            \hline
            \textbf{Имя столбца} & \textbf{Тип}  & \textbf{Описание}                \\
            \hline
            id                   & BIGSERIAL     & Уникальный идентификатор покупки \\
            \hline
            customer\_id         & INT           & Ссылка на клиента (customer.id)  \\
            \hline
            shop\_id             & INT           & Ссылка на магазин (shop.id)      \\
            \hline
            purchase\_date       & TIMESTAMP     & Дата покупки                     \\
            \hline
            total\_amount        & NUMERIC(10,2) & Общая сумма покупки              \\
            \hline
        \end{tabular}
    \end
    Используем BIGSERIAL для подстраховки от переполнения номеров покупок.

    \subsubsection{Таблица purchase\_item}

    \begin
        \normalsize
        \renewcommand{\arraystretch}{1.5} % Регулируем высоту строк
        \begin{tabular}{|l|l|l|}
            \hline
            \textbf{Имя столбца} & \textbf{Тип} & \textbf{Описание}               \\
            \hline
            purchase\_id         & BIGINT       & Ссылка на покупку (purchase.id) \\
            \hline
            product\_id          & INT          & Ссылка на продукт (product.id)  \\
            \hline
            quantity             & INT          & Количество товара в покупке     \\
            \hline
        \end{tabular}
    \end

    \subsubsection{Индексы}

    \begin{center}
        \resizebox{\textwidth}{!}{%
            \begin{tabular}{|l|l|l|}
                \hline
                \textbf{Имя индекса}          & \textbf{Таблица} & \textbf{Описание}                                                           \\
                \hline
                idx\_product\_search          & product          & Индекс для поиска по описанию и характеристикам продукта (используется GIN) \\
                \hline
                idx\_product\_category        & product          & Индекс по категории продукта (category\_id)                                 \\
                \hline
                idx\_inventory\_product\_shop & inventory        & Индекс по продуктам и магазинам в инвентаре                                 \\
                \hline
                idx\_purchase\_customer\_date & purchase         & Индекс по покупателю и дате покупки                                         \\
                \hline
                idx\_purchase\_brin           & purchase         & Индекс с использованием BRIN для диапазона дат покупок                      \\
                \hline
                idx\_customer\_phone          & customer         & Уникальный индекс по телефону клиента                                       \\
                \hline
            \end{tabular}
        }
    \end{center}

    \subsection{Пользовательские типы данных}

    В данной секции приведены пользовательские типы и домены, используемые в базе данных.

    \begin{itemize}
        \item \textbf{email\_domain} – текстовый тип, содержащий email-адрес. Соответствует регулярному выражению:
        \begin{verbatim}
    ^[A-Za-z0-9._%+-]+@[A-Za-z0-9.-]+\.[A-Za-z]{2,}$
        \end{verbatim}

        \item \textbf{phone\_domain} – текстовый тип для хранения телефонных номеров. Допустимые значения:
        \begin{verbatim}
    ^\+?\d{10,15}$
        \end{verbatim}

        \item \textbf{loyalty\_status} – перечислимый тип, определяющий уровень лояльности клиента. Возможные значения:
        \begin{center}
            \begin{tabular}{|c|c|}
                \hline
                Значение & Описание        \\
                \hline
                BRONZE   & Базовый уровень \\
                SILVER   & Средний уровень \\
                GOLD     & Высший уровень  \\
                \hline
            \end{tabular}
        \end{center}
    \end{itemize}

    \subsection{Процедуры, функции и задания по расписанию}

    В данной секции приведены хранимые процедуры, функции и задания, выполняемые в базе данных.

    \subsubsection{Процедуры}

    \begin{itemize}
        \item \textbf{archive\_old\_purchases} – процедура для архивации устаревших данных о покупках.
        \begin{itemize}
            \item Выполняет перенос устаревших записей в архивную таблицу.
            \item Освобождает основную таблицу от старых данных.
        \end{itemize}
    \end{itemize}

    \subsubsection{Функции}

    \begin{itemize}
        \item \textbf{transliterate} – функция для транслитерации текста.
        \begin{itemize}
            \item Принимает строку на входе.
            \item Возвращает строку, в которой символы заменены на их латинские аналоги.
        \end{itemize}
    \end{itemize}
    Функция может пригодиться при миграциях и расширении БД.
    В данном проекте она нашла применение для эстетичности генерируемых данных :)

    \subsubsection{Задание по расписанию}

    В базе используется планировщик задач для автоматического выполнения архивации старых покупок.

    \begin{itemize}
        \item \textbf{Задание архивации:} выполняется ежедневно в 04:00.
    \end{itemize}

    \begin{lstlisting}[language=SQL, frame=single, basicstyle=\normalsize\ttfamily, breaklines=true]
    SELECT cron.schedule('0 4 * * *',
        $$CALL business.archive_old_purchases();$$);
    \end{lstlisting}

    \subsection{Схема cron: описание}
    Данная схема создана подключенным в ходе инициализации БД расширением pg\_cron и содержит технические таблицы
    job и job\_run\_details с информацией о заданиях.

    \subsection{Схема migrations: описание}
    Данная схема создана используемым для управления миграциями инструментом flyway и содержит
    единственную техническую таблицу flyway\_schema\_history с информацией о миграциях.
    
    \section{Кластер Patroni и настройки узлов БД}
    Опишем настройки\ldots

\end{document}
