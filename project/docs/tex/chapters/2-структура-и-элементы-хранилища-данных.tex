\subsection{Схемы}\label{subsec:schemas}
В базе созданы следующие схемы:
\begin{itemize}
    \item \textbf{archive}: архивированные данные, для разгрузки оперативной схемы.
    \item \textbf{business}: оперативная схема с данными о покупках в сети зоомагазинов.
    \item \textbf{cron}: техническая схема для элементов расширения cron.
    \item \textbf{migrations}: техническая схема инструмента миграции flyway.
    \item \textbf{public}: общая начальная схема PostgreSQL\@.
\end{itemize}

\subsection{Схема archive: описание}\label{subsec:archive}

Архивируются только данные о покупках, как наиболее тяжёлые и интенсивные.
Структура и индексы полностью дублируют прототипы из бизнес-схемы.

\subsection{Схема business: описание}\label{subsec:business}

Оперативные бизнес-данные о покупках, магазинах, покупателях.

\begin{figure}[h]
    \centering
    \resizebox{\textwidth}{!}{%
        \begin{tikzpicture}[
            node distance=2cm and 3cm,
            every node/.style={draw, rectangle, rounded corners, minimum width=3cm, minimum height=1cm, align=center}
        ]
            % Определяем таблицы
            \node (category) {category};
            \node (product) [below=of category] {product};
            \node (inventory) [left=of product] {inventory};
            \node (shop) [below=of inventory] {shop};
            \node (supplier) [right=of product] {supplier};
            \node (purchase) [below=of product] {purchase};
            \node (purchase_item) [right=of purchase] {purchase\_item};
            \node (customer) [below=of purchase] {customer};

            % Связи между таблицами
            \draw[->] (product) -- (category);
            \draw[->] (product) -- (supplier);
            \draw[->] (inventory) -- (shop);
            \draw[->] (inventory) -- (product);
            \draw[->] (purchase) -- (customer);
            \draw[->] (purchase) -- (shop);
            \draw[->] (purchase_item) -- (purchase);
            \draw[->] (purchase_item) -- (product);
        \end{tikzpicture}
    }
    \caption{Схема связей таблиц}\label{fig:figure-tables}
\end{figure}

\subsubsection{Таблица category}

\begin
    \normalsize
    \renewcommand{\arraystretch}{1.5} % Регулируем высоту строк
    \begin{tabular}{|l|l|l|}
        \hline
        \textbf{Имя столбца} & \textbf{Тип} & \textbf{Описание}                  \\
        \hline
        id                   & SERIAL       & Уникальный идентификатор категории \\
        \hline
        name                 & TEXT         & Название категории                 \\
        \hline
    \end{tabular}
\end

\subsubsection{Таблица product}

\begin
    \normalsize
    \renewcommand{\arraystretch}{1.5} % Регулируем высоту строк
    \begin{tabular}{|l|l|l|}
        \hline
        \textbf{Имя столбца} & \textbf{Тип}  & \textbf{Описание}                      \\
        \hline
        id                   & SERIAL        & Уникальный идентификатор продукта      \\
        \hline
        name                 & TEXT          & Название продукта                      \\
        \hline
        description          & TEXT          & Описание продукта                      \\
        \hline
        category\_id         & INT           & Ссылка на категорию (category.id)      \\
        \hline
        price                & NUMERIC(10,2) & Цена продукта                          \\
        \hline
        characteristics      & JSONB         & Характеристики продукта в формате JSON \\
        \hline
    \end{tabular}
\end

\subsubsection{Таблица shop}

\begin
    \normalsize
    \renewcommand{\arraystretch}{1.5} % Регулируем высоту строк
    \begin{tabular}{|l|l|l|}
        \hline
        \textbf{Имя столбца} & \textbf{Тип} & \textbf{Описание}                 \\
        \hline
        id                   & SERIAL       & Уникальный идентификатор магазина \\
        \hline
        name                 & TEXT         & Название магазина                 \\
        \hline
        location             & TEXT         & Местоположение магазина           \\
        \hline
    \end{tabular}
\end

\subsubsection{Таблица inventory}

\begin
    \normalsize
    \renewcommand{\arraystretch}{1.5} % Регулируем высоту строк
    \begin{tabular}{|l|l|l|}
        \hline
        \textbf{Имя столбца} & \textbf{Тип} & \textbf{Описание}              \\
        \hline
        shop\_id             & INT          & Ссылка на магазин (shop.id)    \\
        \hline
        product\_id          & INT          & Ссылка на продукт (product.id) \\
        \hline
        quantity             & INT          & Количество товара в магазине   \\
        \hline
    \end{tabular}
\end

\subsubsection{Таблица supplier}

\begin
    \normalsize
    \renewcommand{\arraystretch}{1.5} % Регулируем высоту строк
    \begin{tabular}{|l|l|l|}
        \hline
        \textbf{Имя столбца} & \textbf{Тип} & \textbf{Описание}                   \\
        \hline
        id                   & SERIAL       & Уникальный идентификатор поставщика \\
        \hline
        name                 & TEXT         & Название поставщика                 \\
        \hline
        contact\_info        & TEXT         & Контактная информация поставщика    \\
        \hline
    \end{tabular}
\end

\subsubsection{Таблица customer}

\begin
    \normalsize
    \renewcommand{\arraystretch}{1.5} % Регулируем высоту строк
    \begin{tabular}{|l|l|l|}
        \hline
        \textbf{Имя столбца} & \textbf{Тип}    & \textbf{Описание}                  \\
        \hline
        id                   & SERIAL          & Уникальный идентификатор клиента   \\
        \hline
        phone                & phone\_domain   & Телефон клиента                    \\
        \hline
        email                & email\_domain   & Электронная почта клиента          \\
        \hline
        name                 & TEXT            & Имя клиента                        \\
        \hline
        loyalty\_status      & loyalty\_status & Статус лояльности клиента          \\
        \hline
        bonus\_points        & NUMERIC(10,2)   & Количество бонусных баллов клиента \\
        \hline
    \end{tabular}
    \vspace{10pt} % или \bigskip, если нужно побольше
\end

Несмотря на предполагаемую валидацию на бэкенде, правильные базовые типы в целом в базе не помешают.
Перечисление поможет избежать случайных описок и логически ограничит значения.

\subsubsection{Таблица purchase}

\begin
    \normalsize
    \renewcommand{\arraystretch}{1.5} % Регулируем высоту строк
    \begin{tabular}{|l|l|l|}
        \hline
        \textbf{Имя столбца} & \textbf{Тип}  & \textbf{Описание}                \\
        \hline
        id                   & BIGSERIAL     & Уникальный идентификатор покупки \\
        \hline
        customer\_id         & INT           & Ссылка на клиента (customer.id)  \\
        \hline
        shop\_id             & INT           & Ссылка на магазин (shop.id)      \\
        \hline
        purchase\_date       & TIMESTAMP     & Дата покупки                     \\
        \hline
        total\_amount        & NUMERIC(10,2) & Общая сумма покупки              \\
        \hline
    \end{tabular}
    \vspace{10pt} % или \bigskip, если нужно побольше
\end

Используем BIGSERIAL для подстраховки от переполнения номеров покупок.

\subsubsection{Таблица purchase\_item}

\begin
    \normalsize
    \renewcommand{\arraystretch}{1.5} % Регулируем высоту строк
    \begin{tabular}{|l|l|l|}
        \hline
        \textbf{Имя столбца} & \textbf{Тип} & \textbf{Описание}               \\
        \hline
        purchase\_id         & BIGINT       & Ссылка на покупку (purchase.id) \\
        \hline
        product\_id          & INT          & Ссылка на продукт (product.id)  \\
        \hline
        quantity             & INT          & Количество товара в покупке     \\
        \hline
    \end{tabular}
\end

\subsubsection{Индексы}

\begin{center}
    \resizebox{\textwidth}{!}{%
        \begin{tabular}{|l|l|l|}
            \hline
            \textbf{Имя индекса}          & \textbf{Таблица} & \textbf{Описание}                                                           \\
            \hline
            idx\_product\_search          & product          & Индекс для поиска по описанию и характеристикам продукта (используется GIN) \\
            \hline
            idx\_product\_category        & product          & Индекс по категории продукта (category\_id)                                 \\
            \hline
            idx\_inventory\_product\_shop & inventory        & Индекс по продуктам и магазинам в инвентаре                                 \\
            \hline
            idx\_purchase\_customer\_date & purchase         & Индекс по покупателю и дате покупки                                         \\
            \hline
            idx\_purchase\_brin           & purchase         & Индекс с использованием BRIN для диапазона дат покупок                      \\
            \hline
            idx\_customer\_phone          & customer         & Уникальный индекс по телефону клиента                                       \\
            \hline
        \end{tabular}
    }
\end{center}

\subsection{Пользовательские типы данных}\label{subsec:usertypes}

В данной секции приведены пользовательские типы и домены, используемые в базе данных.

\begin{itemize}
    \item \textbf{email\_domain} – текстовый тип, содержащий email-адрес.
    Соответствует регулярному выражению:
    \begin{verbatim}
^[A-Za-z0-9._%+-]+@[A-Za-z0-9.-]+\.[A-Za-z]{2,}$
    \end{verbatim}

    \item \textbf{phone\_domain} – текстовый тип для хранения телефонных номеров.
    Допустимые значения:
    \begin{verbatim}
^\+?\d{10,15}$
    \end{verbatim}

    \item \textbf{loyalty\_status} – перечислимый тип, определяющий уровень лояльности клиента.
    Возможные значения:
    \begin{center}
        \begin{tabular}{|c|c|}
            \hline
            Значение & Описание        \\
            \hline
            BRONZE   & Базовый уровень \\
            SILVER   & Средний уровень \\
            GOLD     & Высший уровень  \\
            \hline
        \end{tabular}
    \end{center}
\end{itemize}

\subsection{Процедуры, функции и задания по расписанию}\label{subsec:procedures}

В данной секции приведены хранимые процедуры, функции и задания, выполняемые в базе данных.

\subsubsection{Процедуры}

\begin{itemize}
    \item \textbf{archive\_old\_purchases} – процедура для архивации устаревших данных о покупках.
    \begin{itemize}
        \item Выполняет перенос устаревших записей в архивную таблицу.
        \item Освобождает основную таблицу от старых данных.
    \end{itemize}
\end{itemize}

\subsubsection{Функции}

\begin{itemize}
    \item \textbf{transliterate} – функция для транслитерации текста.
    \begin{itemize}
        \item Принимает строку на входе.
        \item Возвращает строку, в которой символы заменены на их латинские аналоги.
    \end{itemize}
\end{itemize}

Функция может пригодиться при миграциях и расширении БД.
В данном проекте она нашла применение для эстетичности генерируемых данных :)

\subsubsection{Задание по расписанию}

В базе используется планировщик задач для автоматического выполнения архивации старых покупок.

\begin{itemize}
    \item \textbf{Задание архивации:} выполняется ежедневно в 04:00.
\end{itemize}

\begin{lstlisting}[language=SQL, frame=single, basicstyle=\normalsize\ttfamily, breaklines=true,label={lst:cronsql}]
SELECT cron.schedule('0 4 * * *',
    $$CALL business.archive_old_purchases();$$);
\end{lstlisting}

\subsection{Схема cron: описание}\label{subsec:crondesc}

Данная схема создана подключенным в ходе инициализации БД расширением pg\_cron и содержит технические таблицы
`job` и `job\_run\_details` с информацией о заданиях.

\subsection{Схема migrations: описание}\label{subsec:migrationsdesc}

Данная схема создана используемым для управления миграциями инструментом flyway и содержит
единственную техническую таблицу `flyway\_schema\_history` с информацией о миграциях.