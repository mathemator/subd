Учебный проект \textbf{Лап-Ландия} посвящён построению базы данных для зоомагазина с развёрнутой инфраструктурой,
включающей реплицируемую БД, балансировщик нагрузки,
инструменты мониторинга и анализатора данных.
Демонстрационный стенд разворачивается ``по клику`` через docker compose.

\subsection{Общая структура}\label{subsec:-generalstructure:-}

\begin{itemize}
    \item \textbf{База данных}: кластер Patroni из трёх узлов.
    \item \textbf{Балансировка}: HAProxy (1 демонстрационный экземпляр).
    \item \textbf{Распределённая Система Контроля (DCS)}: Zookeeper 3 экземпляра.
    \item \textbf{Инициализация БД}: мини-контейнер db-script-runner (создаёт базу и юзеров).
    \item \textbf{Миграции}: Flyway.
    \item \textbf{Мониторинг}: PostgreSQL Exporter + Prometheus + Grafana.
    \item \textbf{Бизнес-аналитика}: Apache Superset.
    \item \textbf{Демонстрационное приложение}: Spring Boot (Java).
    \item \textbf{Демонстрационное тестирование API}: контейнер curl\_runner (отправляет запросы в приложение).
    \item \textbf{Фронт (бонус)}: TypeScript (в разработке).
    \item \textbf{Логирование в ClickHouse (бонус)}: через Logstash.
\end{itemize}

Один экземпляр балансировщика HAProxy был развёрнут с пониманием того, что в реальном производстве их должно быть
минимум два.
Но, т.к. всё окружение разворачивается в тестовой среде на персональном компьютере ученика и точка отказа, так или
иначе, одна - работает один экземпляр.
Теоретически можно было бы обойтись двумя узлами Zookeeper и Patroni, но, как и многое в этом проекте, эти настройки
были взяты из открытых источников и, ввиду и без того не самой тривиальной архитектуры стенда, было решено
не тратить время на переделку готовых наработок коллег-ремесленников.

Аналогично не реплицировались Superset, ClickHouse и другие системы, т.к. предпочтительной задачей в контексте курса
была выбрана настройка некоторого одного ``целевого`` кластера СУБД.

\newpage

\subsection{Диаграмма инфраструктуры}\label{subsec:-structurediagram:-}

\begin{figure}[h]
    \centering
    \resizebox{\textwidth}{!}{%
        \begin{tikzpicture}

            \node (app) [draw, rectangle, rounded corners, minimum width=3cm, minimum height=1cm] {Java-приложение};
            \node (haproxy) [draw, rectangle, rounded corners, minimum width=3cm, minimum height=1cm, below=of app] {HAProxy};
            \node (db1) [draw, rectangle, rounded corners, minimum width=3cm, minimum height=1cm, below left=of haproxy] {Patroni 1};
            \node (db2) [draw, rectangle, rounded corners, minimum width=3cm, minimum height=1cm, below=of haproxy] {Patroni 2};
            \node (db3) [draw, rectangle, rounded corners, minimum width=3cm, minimum height=1cm, below right=of haproxy] {Patroni 3};
            \node (zookeeper) [draw, rectangle, below=of db2, minimum width=3cm, minimum height=1cm] {Zookeeper (3 экз.)};
            \node (flyway) [draw, rectangle, rounded corners, minimum width=3cm, minimum height=1cm, below=of db1] {Flyway, db-init};
            \node (superset) [draw, rectangle, rounded corners, minimum width=3cm, minimum height=1cm, right=of haproxy] {Superset};
            \node (logstash) [draw, rectangle, rounded corners, minimum width=3cm, minimum height=1cm, above=of app] {LogStash};
            \node (front) [draw, rectangle, rounded corners, minimum width=3cm, minimum height=1cm, right=of app] {Front/Curl};
            \node (clickhouse) [draw, rectangle, rounded corners, minimum width=3cm, minimum height=1cm, above right=of app] {ClickHouse};
            \node (pgexporter) [draw, rectangle, rounded corners, minimum width=3cm, minimum height=1cm, left=of haproxy] {PG-exporter};
            \node (prometheus) [draw, rectangle, rounded corners, minimum width=3cm, minimum height=1cm, above=of pgexporter] {Prometheus};
            \node (grafana) [draw, rectangle, rounded corners, minimum width=3cm, minimum height=1cm, above=of prometheus] {Grafana};

            \draw[->] (app) -- (haproxy);
            \draw[->] (superset) -- (haproxy);
            \draw[->] (haproxy) -- (db1);
            \draw[->] (haproxy) -- (db2);
            \draw[->] (haproxy) -- (db3);
            \draw[->] (zookeeper) -- (db1);
            \draw[->] (zookeeper) -- (db2);
            \draw[->] (zookeeper) -- (db3);
            \draw[->, dash pattern=on 2pt off 2pt] (flyway) -- (haproxy)
            \draw[->] (logstash) -- (clickhouse);
            \draw[->] (app) -- (logstash);
            \draw[->] (front) -- (app);
            \draw[->] (pgexporter) -- (haproxy);
            \draw[->] (prometheus) -- (pgexporter);
            \draw[->] (grafana) -- (prometheus);
        \end{tikzpicture}
    }
    \caption{Схема инфраструктуры проекта}\label{fig:structurefigure}
\end{figure}