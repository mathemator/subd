В демонстрационных целях было разработано веб-приложение, работающее с базами данных через актуальный подход
Spring-Boot+JPA и Hibernate.

\subsection{API приложения}\label{subsec:api-server-app}
В рамках демонстрационных задач в приложении объявлены эндпойнты по:
\begin{itemize}
    \item Получению базовой информации по продукту.
    \item Получению информации о покупках в рамках пагинации (через Slice).
    \item Получению информации о продуктах в рамках пагинации (через Page).
    \item Совершению случайной покупки - для презентации вставки данных в БД.
\end{itemize}

\subsection{Поддержка транзакционности в приложении}\label{subsec:apptransaction}
В серверном приложении используется механизм транзакций Spring для обеспечения целостности данных.
Методы, выполняющие критически важные операции, такие как совершение покупки, аннотированы @Transactional,
что позволяет автоматически управлять началом, фиксацией и откатом транзакции.
По умолчанию в Spring используется уровень изоляции \textbf{READ\_COMMITED},
который гарантирует, что транзакция не видит неподтверждённых изменений других транзакций,
предотвращая ``грязные`` чтения.
Такой подход обеспечивает корректность выполнения бизнес-логики, сохраняя консистентность данных в базе.


\subsection{Пагинация в контексте работы с БД}\label{subsec:jpapagination}

\subsubsection{Пагинация по умолчанию}\label{subsubsec:defaultpagination}
В базовой пагинации, реализованной в Spring Boot Jpa, как и во многих других фреймворках,
в частности, Django в Python - используется извлечение через offset с запросом общего числа данных в таблице.
Это иллюстрируют записи в логе приложения от Hibernate, при выполнении запроса по продуктам:

\begin{lstlisting}[language=bash, frame=single, basicstyle=\normalsize\ttfamily, breaklines=true,label={lst:hiberpagelog}]
2025-03-02 13:23:22 Hibernate: select p1_0.id,p1_0.category_id,p1_0.characteristics,p1_0.description,p1_0.name,p1_0.price from business.product p1_0 offset ? rows fetch first ? rows only
2025-03-02 13:23:22 Hibernate: select count(p1_0.id) from business.product p1_0
\end{lstlisting}

На что можно обратить внимание:
\begin{itemize}
    \item При больших таблицах \textbf{offset} будет влиять на утилизацию ресурсов, т.к. БД будет арифметически
    идти по таблице, пока не придёт указателем на нужное место.
    \item Имеем дополнительный запрос на число записей, который не во всех сценариях ``прокрутки`` нужен клиенту.
\end{itemize}

\subsubsection{Использование keyset и slice}\label{subsubsec:slicekeysetpagination}
Можно использовать альтернативный запрос с использованием Slice вместо Page - тогда не будут производиться запросы
\textbf{count}, а также вместо offset использовать условие на какое-нибудь индексированное, удобное для упорядочивания
поле.
Например, числовой `id`.
В коде приложения это выглядит как код на HQL:
\begin{figure}[H]
\begin{lstlisting}[language=java, frame=single, keepspaces=true, showstringspaces=false, basicstyle=\normalsize\ttfamily, breaklines=true,label={lst:hqlquery}]
    @Query("""
            from Order p
              join fetch p.customer c
            where p.id > :lastId
            order by p.id asc
            limit :size
            """)
    Slice<Order> findByIdGreaterThanOrderByIdAsc(Integer lastId, int size);
\end{lstlisting}\label{fig:hqlkeyfetchlisting}
\end{figure}
Заметим, что благодаря использованию join fetch была устранена известная \textbf{N+1} проблема при использовании ORM,
когда при работе с связанными сущностями производятся дополнительные зопросы к БД для каждой сущности.

Теперь в логе видим, без offset и count:
\begin{figure}[H]
\begin{lstlisting}[language=bash, frame=single, basicstyle=\normalsize\ttfamily, breaklines=true,label={lst:hiberpagelog}]
2025-03-02 14:48:54 Hibernate: select p1_0.id,p1_0.customer_id,c1_0.id,c1_0.bonus_points,c1_0.email,c1_0.loyalty_status,c1_0.name,c1_0.phone,p1_0.order_date,p1_0.shop_id,p1_0.total_amount from business.order p1_0 join business.customer c1_0 on c1_0.id=p1_0.customer_id where p1_0.id>? order by p1_0.id fetch first ? rows only
\end{lstlisting}\label{fig:hibernooffsetlisting}
\end{figure}

