\subsection{Кластер Patroni}\label{subsec:patroni}

Кластер Patroni разворачивается благодаря подготовленному образу,
включающему в себя установку различных python-библиотек и расширений и настроечный
файл patroni.yml, содержащий различные настройки для Распределённой Системы Контроля (DCS),
в качестве которой был выбран Zookeeper.
В целом, настройки любой DCS практически идентичны и подкрепляются соответствующими библиотеками python.

\begin{lstlisting}[language=YAML, frame=single, basicstyle=\normalsize\ttfamily, breaklines=true,label={lst:patroniyaml}]
bootstrap:
  dcs:
    ttl: 30
...
zookeeper:
  hosts:
    - zoo1:2181
    - zoo2:2181
    - zoo3:2181
  scope: patroni
  namespace: /service/patroni
\end{lstlisting}

\subsection{Распределённый контроль с помощью Zookeeper}\label{subsec:zookeeper}

DCS была выбрана исходя из наличия минимального знакомства с ней благодаря работе с Apache Kafka.
Zookeeper поддерживает иерархичное храрение ключей и является устоявшейся системой, и хорошо подходит для
хранения не часто меняющихся данных.
При выборе продуктового варианта, однако, можно рассмотреть выбор и ETCD, популярность которого растёт на фоне,
как принято считать, отличается более простым настраиванием и интеграцией с kubernetes.
В самом Zookeeper можно поинтересоваться данными patroni, такими как список узлов, конфигурацией, переданной для
DCS, или, например, кто сейчас является лидером:

\begin{lstlisting}[language=bash, frame=single, basicstyle=\normalsize\ttfamily, breaklines=true,label={lst:zkclish}]
$ root@zoo1:/zookeeper-3.4.14/bin# zkCli.sh
Connecting to localhost:2181
...
get /service/patroni/leader
{"leader": "patroni1", "epoch": 0}
...
\end{lstlisting}

\subsection{Балансировщик с высокой доступностью HAProxy}\label{subsec:haproxy}

HAProxy настроен на API нашего Patroni так, что healthcheck проверяет HTTP-статус 200.
Такой статус возвращает только мастер-узел, а реплики возвращают 503.

Однако, при желании или необходимости, можно разгрузить мастер-узел от чтений, добавив в HAProxy дополнительный
backend.
Это потребует от нас развить работающие с БД приложения на использование двух источников данных с балансировкой,
например, в контексте инструментария Spring Boot JPA\@.
В данной работе этот вариант только обозначим, а настроим на прямую работу только с мастером.