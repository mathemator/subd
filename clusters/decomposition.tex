\documentclass[a4paper,12pt]{article}
\usepackage[T2A]{fontenc}
\usepackage[utf8]{inputenc}
\usepackage[russian]{babel}
\usepackage{graphicx}
\usepackage{amsmath, amssymb}
\usepackage{geometry}
\usepackage{tikz}
\usetikzlibrary{positioning}

\geometry{left=2cm, right=2cm, top=2cm, bottom=2cm}


\title{Декомпозиция и нормализация данных на примере учебного файла}

\author{Александр Гоппе}

\date{\today}



\begin{document}

    \maketitle

    \section{Исходная структура данных}

    Исходные данные представлены в плоской структуре, где атрибуты клиента и его адреса объединены в одной таблице:

    \begin{center}
        \resizebox{\textwidth}{!}{%
        \begin{tabular}{|c|c|c|c|c|c|c|c|c|c|c|c|c|}
            \hline
            title & first\_name & last\_name & language & birth\_date & gender & marital\_status & country & postal\_code & region & city & street & building \\
            \hline
            Dr. & Danilo & Ambrosini & IT & 1900-01-01 & Unknown & "" & IT & 21100 & "" & Varese & Via Dolomiti & 13 \\
            Mrs & Deborah & Thomas & EN & "" & Female & "" & GB & RH16 3TQ & "" & Haywards Heath & Cobbetts Mead & 31 \\
            \hline
        \end{tabular}%
        }
    \end{center}

    \section{Проблемы исходной структуры}

    \begin{itemize}
        \item Отсутствие типизации (все данные хранятся как строки).
        \item Отсутствие нормализации (данные о клиенте и адресе объединены).
        \item Повторяющиеся данные (страны, города, улицы, языки, пол).
        \item Затрудненная модификация и поддержка (например, изменение названия города потребует обновления всех записей).
    \end{itemize}

    \section{Нормализация и конечная модель данных}

    В результате декомпозиции были выделены следующие сущности:

    \begin{itemize}
        \item \textbf{customers} (id, title\_id, first\_name, last\_name, birth\_date, gender\_id, marital\_status\_id, language\_id)
        \item \textbf{titles} (id, name)
        \item \textbf{genders} (id, name)
        \item \textbf{languages} (id, name)
        \item \textbf{marital\_statuses} (id, name)
        \item \textbf{countries} (id, name, code)
        \item \textbf{cities} (id, name, region, country\_id)
        \item \textbf{streets} (id, name, city\_id)
        \item \textbf{addresses} (id, postal\_code, street\_id, building\_number)
        \item \textbf{customer\_addresses} (customer\_id, address\_id)
    \end{itemize}

    \subsection{Диаграмма конечной структуры}
    \begin{center}
        \begin{tikzpicture}[node distance=1.5cm, every node/.style={draw, minimum width=3cm, align=center}]
            \node (customers) {customers};
            \node (titles) [above left=of customers] {titles};
            \node (genders) [left=of customers] {genders};
            \node (languages) [above right=of customers] {languages};
            \node (marital) [right=of customers] {marital\_statuses};
            \node (addresses) [below=of customers] {addresses};
            \node (customer_addresses) [below=of addresses] {customer\_addresses};
            \node (countries) [below right=of addresses] {countries};
            \node (cities) [above right=of countries] {cities};
            \node (streets) [above=of countries] {streets};

            \draw[->] (customers) -- (titles);
            \draw[->] (customers) -- (genders);
            \draw[->] (customers) -- (languages);
            \draw[->] (customers) -- (marital);
            \draw[->] (customers) -- (customer_addresses);
            \draw[->] (customer_addresses) -- (addresses);
            \draw[->] (addresses) -- (streets);
            \draw[->] (streets) -- (cities);
            \draw[->] (cities) -- (countries);
        \end{tikzpicture}
    \end{center}


    \section{Заключение}

    Благодаря нормализации удалось:
    \begin{itemize}
        \item Устранить избыточность данных.
        \item Обеспечить целостность и удобство обновления.
        \item Улучшить читаемость структуры и её расширяемость.
    \end{itemize}

\end{document}
