Проектируемая база данных зоомагазина соответствует принципам нормализации,
что позволяет минимизировать избыточность данных, устранить аномалии обновления и обеспечить целостность информации.
Рассмотрим, как наша модель удовлетворяет требованиям нормальных форм.

\subsection{Первая нормальная форма (1NF)}\label{subsec:1nf}
Первая нормальная форма требует, чтобы каждая таблица содержала только атомарные значения,
а каждая строка была уникальной.
\begin{itemize}
    \item Все атрибуты содержат неделимые значения (например, `phone` и `email` в таблице `customer`).
    \item В таблице `inventory` используется составной первичный ключ `(shop\_id, product\_id)`,
    что исключает дублирование записей о наличии товаров в магазинах.
    \item В таблице `purchase\_item` связь между покупкой и товарами организована через `purchase\_id` и `product\_id`,
    что позволяет хранить любое количество товаров в одной покупке без нарушения 1NF\@.
\end{itemize}

\subsection{Вторая нормальная форма (2NF)}\label{subsec:2nf}
Вторая нормальная форма требует выполнения 1NF и отсутствия частичных зависимостей от составного ключа.
\begin{itemize}
    \item Все таблицы, кроме `inventory` и `purchase\_item`, имеют простой первичный ключ.
    \item В `inventory` и `purchase\_item` данные зависят от обоих ключей в составном первичном ключе,
    что соответствует 2NF\@.
    \item Атрибуты зависят исключительно от идентификаторов записей, например,
    `loyalty\_status` в `customer` и `price` в `product`.
\end{itemize}

\subsection{Третья нормальная форма (3NF)}\label{subsec:3nf}
Третья нормальная форма требует, чтобы в таблице не было транзитивных зависимостей.
\begin{itemize}
    \item В таблице `customer` атрибут `loyalty\_status` является перечисляемым типом (`ENUM`),
    что предотвращает избыточность данных.
    \item Поля `category\_id` и `supplier\_id` в `product` хранят только идентификаторы,
    а полные сведения находятся в соответствующих таблицах.
    \item В `purchase` `total\_amount` хранится явно, но он не зависит от `customer\_id` или `shop\_id`,
    а вычисляется из `purchase\_item`, что не нарушает 3NF\@.
\end{itemize}

\subsection{Дополнительные аспекты нормализации}\label{subsec:normalizationadditional}
\begin{itemize}
    \item Использование доменных типов (`email\_domain`, `phone\_domain`)
    не только упрощает контроль данных, но и способствует нормализации, предотвращая дублирование проверок.
    \item JSONB-колонка `characteristics` в `product` допускает хранение специфических
    характеристик товаров без нарушения принципов нормализации, так как JSONB-данные не
    участвуют в ключах или зависимостях.
    \item Ограничения целостности (`CHECK`, `FOREIGN KEY`, `UNIQUE`) обеспечивают
    консистентность данных на уровне СУБД.
\end{itemize}

\subsection{Вывод}\label{subsec:normalizationresult}
Данная модель базы данных соответствует третьей нормальной форме (3NF),
обеспечивая структурированное и эффективное хранение данных.
Она балансирует между строгой нормализацией и практичностью, например,
JSONB-атрибуты добавляют гибкость без нарушения основных нормализационных требований.
Такой подход минимизирует дублирование данных, упрощает обновления и поддерживает
высокую производительность запросов.

